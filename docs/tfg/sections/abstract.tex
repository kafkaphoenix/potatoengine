\section*{Resumen}

El mundo de los videojuegos ha demostrado ser, lejos de una moda pasajera, una forma de ocio cada vez más popular, entre jóvenes
y adultos por igual, con un futuro brillante. Esto ha creado un gran interés en todos los aspectos relacionados con 
el desarrollo de juegos, y toda una proliferación de cursos, vídeos, libros y revistas sobre el tópico; así como 
diferentes herramientas gratuitas y de pago para la creación de los mismos.

Este \gls{tfg} tiene un doble objetivo, por un lado ofrecer un motor de juegos de código abierto y gratuito, que permita a los usuarios
crear juegos en \gls{2d-es} o \gls{3d-es}, abstrayéndoles de toda la complejidad subyacente; y una faceta didáctica, explicando
a los usuarios, a través de una serie de videos tutoriales y diagramas, la creación de juegos usando el motor para tal fin.

Durante el desarrollo de este \gls{tfg}, se estudiará la arquitectura necesaria para la elaboración de un motor de 
juegos, y se analizarán las soluciones disponibles con sus ventajas y desventajas. Finalmente, se realizará un prototipo 
del motor de juegos, mostrando diferentes casos de uso del motor, incluyendo un juego de tipo plataforma en \gls{2d-es}.

\section*{Palabras Claves}

Motor de juegos, videojuegos, OpenGL, GLFW, Entidad Componente Sistema

\section*{Abstract}

The world of videogames has proven to be, far from a temporary trend, an increasingly popular hobby, both among the young
and adults, with a foreseeable brilliant future. This has created a surge of interest in the subject of game 
development: courses, videos, books and magazines; as well as free and paid tools for their creation.

This \gls{bfp} serves a dual purpose, firstly to offer a free and open source game engine, that allows the users to
create \gls{2d-en} or \gls{3d-en} games, abstracting them of all the complexity behind it; secondly, from a didactive
perspective explaining to the users about game development using the game engine for it.

During the development of this \gls{bfp}, the required architecture for making a game engine will be studied, taking into account
the present and past solutions, with their advantages and disadvantages. Finally, a prototype of the game engine
will be developed portraying different use cases of the engine, including a platform game in \gls{2d-en}.

\section*{Keywords}

Game engine, videogames, OpenGL, GLFW, Entity Component System