\chapter*{6. Documentación y Curso de formación}\addcontentsline{toc}{chapter}{6. Documentación y Curso de formación}\label{cap:course}

Además de ofrecer el motor de juegos como herramienta, es necesaria la formación de los usuarios en su uso. Para ello,
se han creado dos cursos en formato vídeo, el primero sobre el propio motor que se divide en los siguientes capítulos:
\begin{itemize}
	\item \textbf{Vídeos del motor}:
        \begin{itemize}
            \item Inicialización del motor: Entrypoint, logging y aplicación.
            \item Settings Manager: Carga y guardado.
            \item States y Layers Manager.
            \item Windows Manager.
            \item Render Manager: Programas de shaders y framebuffers.
            \item ECS, Scene Manager, Scene Factory y Entity Factory.
            \item Assets Manager: Interfaz y assets predefinidos.
            \item Flujo del motor y bucle del juego.
        \end{itemize}
    \item \textbf{Vídeos de la herramienta Debugger}:
        \begin{itemize}
            \item Inspector.
            \item Métricas y Logger.
        \end{itemize}
\end{itemize}
El segundo curso emplea el motor para crear un ejemplo de juego, el cual se ha podido ver en el capítulo de validación
página \pageref{cap:validation}, dividiéndose en los siguientes capítulos:
\begin{itemize}
    \item \textbf{Vídeos caso de uso juego 2D}:
        \begin{itemize}
            \item Configuración del entorno: Clonar repositorio, estructura de archivos y carpetas.
            \item Archivos de Prefab y Escena.
            \item Ejemplo estado Menú: Layers.
            \item Ejemplo estado Juego: Layers y Overlays.
            \item Ejemplo estado Juego: Sistemas.
        \end{itemize}
\end{itemize}
Con esta formación se pretende dar al usuario una referencia ante cualquier duda que pueda surgirle desarrollando
un videojuego con el motor. Estos vídeos se acompañan también de diagramas que sirven como referencia rápida.
El usuario puede encontrar el código del juego de demostración dentro del repositorio de Github\cite{project-repository}
pudiéndolo usar como punto de partida en sus propios proyectos.

En el caso de que el usuario decida expandir el motor, todo el desarrollo del motor se ha reflejado en el repositorio
de forma que puede ver commit por commit las funcionalidades que se han ido añadiendo, permitiendo comprender mejor
el motor.