\chapter*{2. Análisis del trabajo relacionado}\addcontentsline{toc}{chapter}{2. Análisis del trabajo relacionado}\label{cap:analysis}

Antes de poder hablar sobre el desarrollo de un motor de juegos, se debe entender qué es un motor y analizar qué ofrecen los principales
motores que se vieron en la \figurename~\ref{engine_distribution}. 

\section*{2.1. ¿Qué es un motor de juegos?}\addcontentsline{toc}{section}{2.1. ¿Qué es un motor de juegos?}\label{sec:game-engine}

Un motor de juegos se compone de diferentes piezas, pero se pueden ver las principales en la \figurename~\ref{engine_components}.

\begin{figure}[h!]
    \centering
    \smartdiagram[constellation diagram]{Motor, Motor Matemático, Motor Gráfico, Motor de Físicas, Motor de Entidades}
    \caption{Componentes de un motor}
    \label{engine_components}
\end{figure}

Lo primero que se necesita para poder crear un juego es crear una ventana donde se pueda \textit{visualizar la escena}, 
de esto se encargará el \textbf{motor gráfico}. Esta escena tendrá elementos que se llamarán \textit{entidades} como puede ser el jugador,
los árboles o los enemigos, de esto se encargará el \textbf{motor de entidades}. Seguidamente esas entidades \textit{reaccionarán al
entorno y entre ellas} a través del \textbf{motor de físicas}. Y por último estos motores necesitarán hacer uso del \textbf{motor matemático}
para realizar las diferentes \textit{operaciones algebraicas} que requieran.

\newpage

\section*{2.2. Conclusiones del análisis y tecnologías empleadas}\addcontentsline{toc}{section}{2.2. Conclusiones del análisis y tecnologías empleadas}\label{sec:technologies}

Se pueden observar en la \tablename~\ref{tab:game-engines-characteristics} las conclusiones de analizar y comparar las tres opciones más populares de motores en la actualidad\cite{comparing-engines} Unity\cite{unity}, Unreal\cite{unreal} y Game Maker\cite{gamemaker}.

\begin{table}[htbp]
    \centering
    \begin{tabular}{@{}|l|
    >{\columncolor[HTML]{DCFF99}}c |
    >{\columncolor[HTML]{DCFF99}}c |
    >{\columncolor[HTML]{DCFF99}}c |@{}}
    \toprule
     & \multicolumn{1}{l|}{\cellcolor[HTML]{FFFFC7}Unity} & \multicolumn{1}{l|}{\cellcolor[HTML]{FFFFC7}Unreal} & \multicolumn{1}{r|}{\cellcolor[HTML]{FFFFC7}Game Maker} \\ \midrule
    \cellcolor[HTML]{FFFFC7}Multi-plataforma         & Sí & Sí & Sí                         \\ \midrule
    \cellcolor[HTML]{FFFFC7}Interfaz Interactiva     & Sí & Sí & Sí                         \\ \midrule
    \cellcolor[HTML]{FFFFC7}Documentación            & Sí & Sí & Sí                         \\ \midrule
    \cellcolor[HTML]{FFFFC7}Plantillas               & Sí & Sí & Sí                         \\ \midrule
    \cellcolor[HTML]{FFFFC7}Juegos 2D                & Sí & Sí & Sí                         \\ \midrule
    \cellcolor[HTML]{FFFFC7}Juegos 3D                & Sí & Sí & \cellcolor[HTML]{FD6864}No \\ \midrule
    \cellcolor[HTML]{FFFFC7}Herramientas de Análisis o Debugger & Sí & Sí & Sí                         \\ \bottomrule
    \end{tabular}
    \caption{Principales características de los motores analizados}
    \label{tab:game-engines-characteristics}
\end{table}

Reuniendo toda la información previa, se tiene una idea más clara de lo que esperaría un usuario de un motor:
\begin{itemize}
    \item Simple de usar.
    \item Extensa Documentación.
    \item Plantillas para obtener una idea inicial del uso del motor.
    \item Herramientas de análisis o \textit{debugger} para mejorar el rendimiento, ver logs e inspeccionar elementos.
    \item Opción de exportar a diferentes plataformas.
    \item Opción de crear juegos tanto en \gls{2d-es} como en \gls{3d-es}.
\end{itemize}
En consecuencia, este proyecto se ha desarrollado usando el lenguaje \textit{C++}, estándar en el desarrollo de
videojuegos por su eficiencia, junto a \textit{CMake}\cite{cmake} que permite la compilación multiplataforma del proyecto.
Además las siguientes librerías se han usado para cubrir las necesidades del motor:
\begin{itemize}
    \item GLFW\cite{glfw} y GLAD\cite{glad}: Librerías para creación de ventanas y gráficos de alto rendimiento usando \textit{OpenGL}\cite{opengl}.
    \item EnTT\cite{entt}: Librería de entidades.
    \item GLM\cite{glm}: Librería de matemáticas.
    \item ImGui\cite{imgui}: Librería para crear la interfaz y menús de la herramienta de análisis o \textit{debugger}.
    \item nlohmann\_json\cite{nlohmann_json}: Carga de plantillas de tipo JSON, para definir escenas y entidades.
    \item stb\cite{stb}: Carga de imágenes.
    \item Assimp\cite{assimp}: Carga de modelos \gls{3d-es}.
    \item spdlog\cite{spdlog}: Logs de la aplicación.
\end{itemize}

\emptyPage